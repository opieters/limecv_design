% !TEX TS-program = xelatex
% !TEX encoding = UTF-8 Unicode

\documentclass[11pt]{article}

% hyphenation 
\usepackage[british]{babel}

% more advanced expressions in \setlength
\usepackage{calc}

% force usage of newer commands (optinonal) 
\usepackage[l2tabu,orthodox]{nag}

% define margin
\newlength\cvMargin
\setlength\cvMargin{1cm}

\usepackage[margin=\cvMargin,noheadfoot,a4paper]{geometry}

% other lengths
\newlength\cvSideWidth
\setlength\cvSideWidth{0.3\paperwidth-\cvMargin}

\newlength\cvMainWidth
\setlength\cvMainWidth{\paperwidth-4\cvMargin-\cvSideWidth}

\newlength\cvPictureWidth
\setlength\cvPictureWidth{4cm}

\newlength\cvLanguageBarWidth
\setlength\cvLanguageBarWidth{5em}

\newlength\cvLanguageBarHeight
\setlength\cvLanguageBarHeight{0.75em}

\newlength\cvTimeDotSep
\setlength\cvTimeDotSep{0.4cm}

\newlength{\cvSectionSep}
\setlength{\cvSectionSep}{0.4cm}

\usepackage{calc}
\newlength\cvHeaderIconWidth

\newcommand{\cvSection}[1]{\Large\textbf{#1}}

\newcommand{\cvEducation}[3]{{\firaMedium #1}\\ #2\\ \emph{#3}}
\newcommand{\cvExperience}[5]{{\firaMedium #1}\\ \textsc{\selectfont #2}, #3. #4\\ \emph{#5}}


% links
\usepackage{hyperref}

% more advanced command definitions
\usepackage{xparse}

% advanced drawing
\usepackage{tikz}
\usetikzlibrary{calc,positioning,backgrounds,matrix,node-families}

% pictures
\usepackage{graphicx}

% define four main colours
\definecolor{cvGreen}{HTML}{357F2D}
\definecolor{cvGreenLight}{HTML}{B8E4B3}
\definecolor{cvDark}{HTML}{2F3142}
\definecolor{cvAccent}{HTML}{474A65}

% load external fonts
\usepackage{fontspec}    

% icon font
\usepackage{fontawesome}   
   
% load external fonts
\setmainfont[Numbers={OldStyle,Monospaced}]{Fira Sans}
\setsansfont{Fira Sans}
\setmonofont{Fira Mono}
\newfontfamily\firaMedium{Fira Sans Medium}

\pagestyle{empty}

% update default paragraph indent, and header space
\setlength{\topskip}{0pt}      % between header and text (0 needed for vertical centring)
\usepackage{parskip}           % remove paragraph indents

% we can only set this length after loading the fonts
\setlength{\cvHeaderIconWidth}{\maxof{\widthof{\faBriefcase}}{\widthof{\faGraduationCap}}}

% set TikZ styles
\tikzset{
  contactIcon/.style={%
    minimum height=\baselineskip,
  },
  contactText/.style={%
    minimum height=\baselineskip,
    text depth=0pt,
  },
  headerIcon/.style={
    minimum width=\cvHeaderIconWidth,
    anchor=center,
  },
  languageText/.style={},
  progressArea/.style={%
    draw,
    rectangle,
    minimum width=\cvLanguageBarWidth,
    minimum height=\cvLanguageBarHeight,
    cvGreen},
  progressBar/.style={%
    minimum height=\cvLanguageBarHeight,
    rectangle,
    draw,
    fill,
    cvGreen,
    anchor=west},
  sectionTitle/.style={
    anchor=north west,
    align=left},
  eventdottext/.style = {
    text width=\cvMainWidth-\cvTimeDotSep,
    black,
    %text depth=0pt,
    anchor=north west,
  },
  sectionedutext/.style={
    eventdottext,
    anchor=north west
  },
  invisibletimedot/.style = {
    circle,
    minimum width=3pt,
    anchor=center
  },
  timedot/.style = {
    invisibletimedot,
    draw,
    minimum width=3pt,
    fill,
    black,
  },
  }

% based on https://tex.stackexchange.com/questions/65731
\makeatletter
\def\cv@hrulefill{{\color{cvGreen}\leavevmode\leaders\hrule height 1pt\hfill\kern\z@}}

% line before and after text (some tweaking is required here)
% based on https://tex.stackexchange.com/questions/15119
\NewDocumentCommand{\ruleline}{m}{\par\noindent\raisebox{\baselineskip/4}{\makebox[\linewidth]{\cv@hrulefill\hspace{1ex}\raisebox{-\baselineskip/4}{#1}\hspace{1ex}\cv@hrulefill}}}
\makeatother

% update global colour
\makeatletter
\NewDocumentCommand{\globalcolor}{m}{%
  \color{#1}\global\let\default@color\current@color
}
\makeatother
\AtBeginDocument{\globalcolor{cvDark}}

\NewDocumentCommand{\cvLanguage}{mm}{%
  {\globaldefs=1\relax\pgfkeys{/cv/languages/lang\the\value{languages} = #2}}
  
  \stepcounter{languages}
}

\makeatletter
\newcount\my@repeat@count
\newcommand{\cvSkill}[1]{%
  \begingroup
  \my@repeat@count=\z@
  \@whilenum\my@repeat@count<#1\do{\faCircle\advance\my@repeat@count\@ne}%
  \my@repeat@count=\numexpr5-\z@\relax
  \@whilenum\my@repeat@count>#1\do{\faCircleO\advance\my@repeat@count\m@ne}%
  \endgroup
}
\makeatother
  
\begin{document}

  \vspace*{\fill}
  \begin{tikzpicture}[remember picture,overlay]
    \fill[cvGreenLight] (current page.north west) rectangle ++(\cvSideWidth+2\cvMargin,-\paperheight);
  \end{tikzpicture}
  \begin{minipage}{\cvSideWidth}
    \begin{center}
  
  \begin{tikzpicture}
    \node[
      circle,
      minimum size=\cvPictureWidth,
      path picture={
        \node at (path picture bounding box.center){
        \includegraphics[width=\cvPictureWidth]{images/profile_picture}
        };
      }]
      {};
  \end{tikzpicture}

  {\LARGE
  John

  \vspace{0.1cm}

  Doe}
  
  \vspace{0.5cm}
  
  {\color{cvAccent} Profession}
  
  \vspace{0.5cm}

  \ruleline{Profile}
  
  Lorem ipsum dolor sit amet, consectetur adipiscing elit. Etiam dictum imperdiet orci, at placerat nulla sagittis id. Praesent iaculis iaculis lorem a aliquam. Nam non fringilla sapien, quis posuere lectus. Quisque ac rhoncus massa. Vestibulum blandit ullamcorper nulla at posuere. In consectetur tempor sem, in interdum mi tempus nec. Cras.
  
  \vspace{4pt}
  
  \ruleline{Contact}
  
  \vspace{4pt}
  
  \begin{tikzpicture}[every node/.style={inner sep=0pt, outer sep=0pt}]
    \matrix [
      column 1/.style={anchor=center,contactIcon},
      column 2/.style={anchor=west,align=left,contactIcon},
      column sep=5pt,
      row sep=5pt] (contact) {
      \node{\faMapMarker}; 
        & \node{Some Street 5\\B-0000 City\\Country};\\
      \node{\faEnvelope}; 
        & \node{\href{mailto:me@johndoe.com}{me@johndoe.com}};\\
      \node{\faPhone}; 
        & \node{+1 781 555 1212};\\
      \node{\faGlobe}; 
        & \node{\href{https://johndoe.com}{johndoe.com}};\\
      \node{\faGithub}; 
        & \node{\href{https://github.com/johndoe}{johndoe}};\\
      \node{\faLinkedinSquare}; 
        & \node{\href{https://www.linkedin.com/in/johndoe/}{johndoe}};\\
      \node{\faTwitter}; 
        & \node{\href{https://twitter.com/JohnDoe}{@JohnDoe}};\\
      \node{\faKey}; 
        & \node{\href{https://keybase.io/johndoe}{\texttt{AAAA BBBB 0000 5555}}};\\
  };
  \end{tikzpicture}
  
  \vspace{4pt}
  
  \ruleline{Languages}
  
  \vspace{4pt}
  
  \begin{tikzpicture}[every node/.style={text depth=0pt,inner sep=0pt,outer sep=0pt}]
    \matrix [
      column 1/.style={anchor=east},
      column 2/.style={anchor=west},
      column sep=5pt,
      row sep=5pt,
      ] (contact) {
      \node[languageText]{English}; & \node[progressArea] (language 1) {}; \\
      \node[languageText]{German};  & \node[progressArea] (language 2) {}; \\
      \node[languageText]{Spanish}; & \node[progressArea] (language 3) {}; \\
    };
    \draw (language 1.west) node[progressBar,minimum width=5/5*\cvLanguageBarWidth] {};
    \draw (language 2.west) node[progressBar,minimum width=3/5*\cvLanguageBarWidth] {};
    \draw (language 3.west) node[progressBar,minimum width=3/5*\cvLanguageBarWidth] {};
  \end{tikzpicture}

    \end{center}
  \end{minipage}
  \vspace*{\fill}
  
\begin{tikzpicture}[
  every node/.style={
    inner sep=0pt,
    outer sep=0pt
  },
  remember picture,
  overlay,
  shift={($(current page.north west)%
           +(\cvSideWidth+3\cvMargin+\cvTimeDotSep,-\cvMargin)$)}]
  % main content  
  
  \node[sectionTitle] at (0,0) (title 1) {\cvSection{Education}};
  \node[left=\cvTimeDotSep of title 1,headerIcon] {\faGraduationCap};
  \begin{scope}[on background layer]
    \draw[line width=2pt,cvGreen] 
      let \p1=(title 1.south west), 
          \p2=(current page.east) in 
      (\x1,\y1-0.1cm) to (\x2,\y1-0.1cm);
  \end{scope}
  
  % item 1
  \node[
      below=\cvSectionSep of title 1.south west,
      eventdottext] 
      (item 1 header) 
      {\phantom{Evening}};
  \node[
    below=\cvSectionSep of title 1.south west,
    sectionedutext] 
    (item 1) 
    {\cvEducation{Evening class: Chinese}%
                 {Some School, City. September 2015 -- June 2016}%
                 {Achieved A2 language skill in Chinese (Mandarin).}};
  \node[
    left=\cvTimeDotSep of item 1 header,
    timedot] 
    (start) 
    {};
    
  % item 2
  \node[
      below=\cvSectionSep of item 1.south west,
      eventdottext] 
      (item 2 header) 
      {\phantom{Evening}};
  \node[
    below=\cvSectionSep of item 1.south west,
    sectionedutext] 
    (item 2) 
    {\cvEducation{Bachelor of Science in Biochemistry and Biotechnology}%
                 {University, City. September 2009 -- June 2012}%
                 {General training in the basic sciences and the molecular life science.}};
  \node[
    left=\cvTimeDotSep of item 2 header,
    timedot] 
    {};
    
  % item 3
  \node[
      below=\cvSectionSep of item 2.south west,
      eventdottext] 
      (item 3 header) 
      {\phantom{Evening}};
  \node[
    below=\cvSectionSep of item 2.south west,
    sectionedutext] 
    (item 3) 
    {\cvEducation{Master of Science in Biochemistry and Biotechnology}%
                 {University, City. September 2012 -- June 2015}%
                 {Acquisition of insight into and knowledge of possibilities for application in the area of biochemistry and biotechnology, specific with applications in biomedical application and due problem-solving reasoning skills.}};
  \node[
    left=\cvTimeDotSep of item 3 header,
    timedot] 
    {};
  \node[
    left=\cvTimeDotSep of item 3.south west,
    invisibletimedot] 
    (end) 
    {};
  \draw (start) to (end.center);
  
  %%%%%%%%%%%%%%
  % Experience %
  %%%%%%%%%%%%%%
  
  \node[below=0.6cm of item 3.south west,sectionTitle] (title 2) {\cvSection{Experience}};
  \node[left=\cvTimeDotSep of title 2,headerIcon] {\faBriefcase};
  \node[below=0.6cm of item 3.south west,sectionTitle] (title 2 dummy) {\phantom{\cvSection{Education}}};
  \begin{scope}[on background layer]
    \draw[line width=2pt,cvGreen] 
      let \p1=(title 2 dummy.south west), 
          \p2=(current page.east) in 
      (\x1,\y1-0.1cm) to (\x2,\y1-0.1cm);
  \end{scope}
  
\node[
      below=\cvSectionSep of title 2.south west,
      eventdottext] 
      (item 1 header) 
      {\phantom{Evening}};
  \node[
    below=\cvSectionSep of title 2.south west,
    sectionedutext] 
    (item 1) 
    {\cvExperience%
      {Student Job}%
      {Company X}%
      {Location X}%
      {Summer 2010}%
      {Integer tincidunt dapibus consectetur. Nullam tristique aliquam luctus. Sed ut ante velit. Nulla pharetra maximus lacus at elementum. Suspendisse sodales consectetur metus, sit amet ultricies ipsum ultrices ut.}};
  \node[
    left=\cvTimeDotSep of item 1 header,
    timedot] 
    (start) 
    {};
    
  % item 2
  \node[
      below=\cvSectionSep of item 1.south west,
      eventdottext] 
      (item 2 header) 
      {\phantom{Evening}};
  \node[
    below=\cvSectionSep of item 1.south west,
    sectionedutext] 
    (item 2) 
    {\cvExperience%
      {Internship}%
      {Company Y}%
      {Location Y}%
      {June 2012 -- August 2012}%
      {Lorem ipsum dolor sit amet, consectetur adipiscing elit. Morbi dictum cursus sapien, id eleifend mi pellentesque id. Etiam lobortis eu odio a sodales. Phasellus ut dolor feugiat, lacinia lectus in, blandit metus. Fusce lacinia dolor et metus gravida pulvinar sit amet et ex.}};
  \node[
    left=\cvTimeDotSep of item 2 header,
    timedot] 
    {};
    
  % item 3
  \node[
      below=\cvSectionSep of item 2.south west,
      eventdottext] 
      (item 3 header) 
      {\phantom{Evening}};
  \node[
    below=\cvSectionSep of item 2.south west,
    sectionedutext] 
    (item 3) 
    {\cvExperience{Internship}%
       {Company Z}%
       {Location Z}%
       {August 2014 -- September 2014}%
       {Lorem ipsum dolor sit amet, consectetur adipiscing elit. Morbi dictum cursus sapien, id eleifend mi pellentesque id. Etiam lobortis eu odio a sodales. Phasellus ut dolor feugiat, lacinia lectus in, blandit metus.  Fusce lacinia dolor et metus gravida pulvinar sit amet et ex. Suspendisse vestibulum, leo malesuada molestie maximus, sem risus ornare elit, vitae sodales felis elit in ipsum.}};
  \node[
    left=\cvTimeDotSep of item 3 header,
    timedot] 
    {};
  \node[
    left=\cvTimeDotSep of item 3.south west,
    invisibletimedot] 
    (end) 
    {};
  \draw (start) to (end.center);

  %%%%%%%%%%
  % Skills %
  %%%%%%%%%%
  
  \node[below=0.6cm of item 3.south west,sectionTitle] (title 3) {\cvSection{Skills}};
  \node[left=\cvTimeDotSep of title 3,headerIcon] {\faStar};
  \node[below=0.6cm of item 3.south west,sectionTitle] (title 3 dummy) {\phantom{\cvSection{Education}}};
  \begin{scope}[on background layer]
    \draw[line width=2pt,cvGreen] 
      let \p1=(title 3 dummy.south west), 
          \p2=(current page.east) in 
      (\x1,\y1-0.1cm) to (\x2,\y1-0.1cm);
  \end{scope}
  
\matrix[matrix of nodes,
        below=0.6cm of title 3.south west,
        anchor=north west,
        column sep=6pt,
        row sep=6pt,
        every node/.style={Text Depth=tdSkills},
        column 1/.style={anchor=east,align=left},
        column 2/.style={anchor=west},
        column 3/.style={anchor=east,align=left},
        column 4/.style={anchor=west}] {
        \cvSkill{5} & MATLAB           & \cvSkill{5} & \LaTeX \\
        \cvSkill{4} & Python           & \cvSkill{4} & VHDL \\
        \cvSkill{4} & Microsoft Office & \cvSkill{4} & macOS \\
        \cvSkill{3} & C, C++           & \cvSkill{3} & Electrical Engineering \\
        \cvSkill{3} & HTML5/CSS        & \cvSkill{3} & Bash \\
        \cvSkill{3} & MPLAB~X          & \cvSkill{2} & Ruby on Rails \\
        \cvSkill{2} & ModelSim         & \cvSkill{1} & Javascript \\};

\end{tikzpicture}

\end{document}